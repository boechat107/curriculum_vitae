%%%%%%%%%%%%%%%%%%%%%%%%%%%%%%%%%%%%%%%%%%%%%%%%%%%%%%%%%%%%%%%%%%%%%%%%
%%%%%%%%%%%%%%%%%%%%%% Simple LaTeX CV Template %%%%%%%%%%%%%%%%%%%%%%%%
%%%%%%%%%%%%%%%%%%%%%%%%%%%%%%%%%%%%%%%%%%%%%%%%%%%%%%%%%%%%%%%%%%%%%%%%

%%%%%%%%%%%%%%%%%%%%%%%%%%%%%%%%%%%%%%%%%%%%%%%%%%%%%%%%%%%%%%%%%%%%%%%%
%% NOTE: If you find that it says                                     %%
%%                                                                    %%
%%                           1 of ??                                  %%
%%                                                                    %%
%% at the bottom of your first page, this means that the AUX file     %%
%% was not available when you ran LaTeX on this source. Simply RERUN  %%
%% LaTeX to get the ``??'' replaced with the number of the last page  %%
%% of the document. The AUX file will be generated on the first run   %%
%% of LaTeX and used on the second run to fill in all of the          %%
%% references.                                                        %%
%%%%%%%%%%%%%%%%%%%%%%%%%%%%%%%%%%%%%%%%%%%%%%%%%%%%%%%%%%%%%%%%%%%%%%%%

%%%%%%%%%%%%%%%%%%%%%%%%%%%%%%%%%%%%%%%%%%%%%%%%%%%%%%%%%%%%%%%%%%%%%%%%
%% NOTE: If you are getting compilation errors referring to list      %%
%%       definitions that don't match, you may need to upgrade to a   %%
%%       newer version of the enumitem package. Try going to:         %%
%%                                                                    %%
%%   http://www.ctan.org/tex-archive/macros/latex/contrib/enumitem    %%
%%                                                                    %%
%%       then download the enumitem.sty file from there. Place it in  %%
%%       the same directory as your CV. So long as there are no other %%
%%       conflicts with older packages on your system, hopefully that %%
%%       will fix your compilation problems.                          %%
%%%%%%%%%%%%%%%%%%%%%%%%%%%%%%%%%%%%%%%%%%%%%%%%%%%%%%%%%%%%%%%%%%%%%%%%

%%%%%%%%%%%%%%%%%%%%%%%%%%%% Document Setup %%%%%%%%%%%%%%%%%%%%%%%%%%%%

% Don't like 10pt? Try 11pt or 12pt
\documentclass[10pt]{article}

% The automated optical recognition software used to digitize resume
% information works best with fonts that do not have serifs. This
% command uses a sans serif font throughout. Uncomment both lines (or at
% least the second) to restore a Roman font (i.e., a font with serifs).
\usepackage[utf8]{inputenc}
\usepackage{times}
\renewcommand{\familydefault}{\sfdefault}

% The OCR software also has a hard time with italics. These commands get
% rid of the two common ways to italicize text in LaTeX. Get rid of them
% to turn italics back on.
\renewcommand\emph[1]{#1}
\renewcommand\textit[1]{#1}

% This is a helpful package that puts math inside length specifications
\usepackage{calc}

% Layout: Puts the section titles on left side of page
\reversemarginpar

%
%         PAPER SIZE, PAGE NUMBER, AND DOCUMENT LAYOUT NOTES:
%
% The next \usepackage line changes the layout for CV style section
% headings as marginal notes. It also sets up the paper size as either
% letter or A4. By default, letter was used. If A4 paper is desired,
% comment out the letterpaper lines and uncomment the a4paper lines.
%
% As you can see, the margin widths and section title widths can be
% easily adjusted.
%
% ALSO: Notice that the includefoot option can be commented OUT in order
% to put the PAGE NUMBER *IN* the bottom margin. This will make the
% effective text area larger.
%
% IF YOU WISH TO REMOVE THE ``of LASTPAGE'' next to each page number,
% see the note about the +LP and -LP lines below. Comment out the +LP
% and uncomment the -LP.
%
% IF YOU WISH TO REMOVE PAGE NUMBERS, be sure that the includefoot line
% is uncommented and ALSO uncomment the \pagestyle{empty} a few lines
% below.
%

%% Use these lines for letter-sized paper
%\usepackage[paper=letterpaper,
%            %includefoot, % Uncomment to put page number above margin
%            marginparwidth=1.2in,     % Length of section titles
%            marginparsep=.05in,       % Space between titles and text
%            margin=1in,               % 1 inch margins
%            includemp]{geometry}

%% Use these lines for A4-sized paper
\usepackage[paper=a4paper,
            %includefoot, % Uncomment to put page number above margin
            marginparwidth=30.5mm,    % Length of section titles
            marginparsep=1.5mm,       % Space between titles and text
            margin=.7in,              % 25mm margins
            includemp]{geometry}

%% More layout: Get rid of indenting throughout entire document
\setlength{\parindent}{0in}

\usepackage[shortlabels]{enumitem}

% Simpler bibsections for CV sections
% (thanks to natbib for inspiration)
%
% * For lists of references with hanging indents and no numbers:
%
%   \begin{bibsection}
%       \item ...
%   \end{bibsection}
%
% * For numbered lists of references (with hanging indents):
%
%   \begin{bibenum}
%       \item ...
%   \end{bibenum}
%
%   Note that bibenum numbers continuously throughout. To reset the
%   counter, use
%
%   \restartlist{bibenum}
%
%   at the place where you want the numbering to reset.

\makeatletter
\newlength{\bibhang}
\setlength{\bibhang}{1em}
\newlength{\bibsep}
 {\@listi \global\bibsep\itemsep \global\advance\bibsep by\parsep}
\newlist{bibsection}{itemize}{3}
\setlist[bibsection]{label=,leftmargin=\bibhang}
\newlist{bibenum}{enumerate}{3}
\setlist[bibenum]{resume,label=[\arabic*]}
\setlist*[bibsection,bibenum]{%
        itemindent=-\bibhang,
        itemsep=\bibsep,parsep=\z@,partopsep=0pt,
        topsep=0pt}
\let\oldendbibenum\endbibenum
\def\endbibenum{\oldendbibenum\vspace{-.6\baselineskip}}
\let\oldendbibsection\endbibsection
\def\endbibsection{\oldendbibsection\vspace{-.6\baselineskip}}
\makeatother

%%% Setup header and footer (with page number and possible last page)
%
% The first block sets up pages 2--end
% The second block sets up page 1 formatting
%
%%%
%
% NOTE: comment the +LP lines and uncomment the -LP lines to have page
%       numbers without the ``of ##'' last page reference)
%
% NOTE: uncomment the \pagestyle{empty} line to get rid of all page
%       numbers on pages 2--end. To get rid of page numbers on page 1,
%       comment out the \thispagestyle{plain} line on the first page
%       below.
%       (also make sure includefoot is commented out above)
%
\usepackage{fancyhdr,lastpage}
\pagestyle{fancy}
%\pagestyle{empty}      % Uncomment this to get rid of page numbers
\fancyhf{}\renewcommand{\headrulewidth}{0pt}
\fancyfootoffset{\marginparsep+\marginparwidth}
\newlength{\footpageshift}
\setlength{\footpageshift}
          {0.5\textwidth+0.5\marginparsep+0.5\marginparwidth-2in}

%%%% PAGES 2--9 NUMBERING:
%% These two lines put page number in upper-right corner of pages 2--end
\rhead{Boechat, p.~\arabic{page} of \protect\pageref*{LastPage}}   % +LP
%\rhead{Pavlic, p.~\arabic{page}}                                 % -LP

%% These lines put page number in bottom (center) of pages 2--end
%\lfoot{\hspace{\footpageshift}%
%       \parbox{4in}{\, \hfill %
%                    \arabic{page} of \protect\pageref*{LastPage} % +LP
%%                    \arabic{page}                               % -LP
%                    \hfill \,}}
%%%% END PAGE 2--9 NUMBERING

%%%% PAGE 1 NUMBERING:
\makeatletter
\let\oldps@plain\ps@plain
\renewcommand{\ps@plain}{\oldps@plain%
\renewcommand{\@evenfoot}{\hspace*{-\footpageshift}\hfil %
    p.~\arabic{page} of \protect\pageref*{LastPage} % +LP
%    p.~\arabic{page}                               % -LP
    \hfil}%
\renewcommand{\@oddfoot}{\@evenfoot}}
\makeatother
%%%% END PAGE 1 NUMBERING

% Finally, give us PDF bookmarks and colored links
%
% NOTE: Some OCR software might be negatively affected by hyperlinks. So
%       most employers recommend the draft option here.
%
% (to enable hyperlinks and bookmarks, comment out ``draft'' line;
%  to disable hyperlinks and bookmarks, uncomment ``draft'' line)
\usepackage{color,hyperref}
\definecolor{darkblue}{rgb}{0.0,0.0,0.3}
\hypersetup{colorlinks,breaklinks,
            linkcolor=darkblue,urlcolor=darkblue,
            anchorcolor=darkblue,citecolor=darkblue,
         %   draft
            }

%%%%%%%%%%%%%%%%%%%%%%%% End Document Setup %%%%%%%%%%%%%%%%%%%%%%%%%%%%


%%%%%%%%%%%%%%%%%%%%%%%%%%% Helper Commands %%%%%%%%%%%%%%%%%%%%%%%%%%%%

%%% HEADING AT TOP OF CURRICULUM VITAE

% The title (name) with a horizontal rule under it
% (optional argument typesets an object right-justified across from name
%  as well)
%
% Usage: \makeheading{name}
%        OR
%        \makeheading[right_object]{name}
%
% Place at top of document. It should be the first thing.
% If ``right_object'' is provided in the square-braced optional
% argument, it will be right justified on the same line as ``name'' at
% the top of the CV. For example:
%
%       \makeheading[\emph{Curriculum vitae}]{Your Name}
%
% will put an emphasized ``Curriculum vitae'' at the top of the document
% as a title. Likewise, a picture could be included:
%
%   \makeheading[\includegraphics[height=1.5in]{my_picutre}]{Your Name}
%
% the picture will be flush right across from the name.
\newcommand{\makeheading}[2][]%
        {\hspace*{-\marginparsep minus \marginparwidth}%
         \begin{minipage}[t]{\textwidth+\marginparwidth+\marginparsep}%
             {\large \bfseries #2 \hfill #1}\\[-0.15\baselineskip]%
                 \rule{\columnwidth}{1pt}%
         \end{minipage}}

%%% SECTION HEADINGS

% The section headings. Flush left in small caps down pseudo-margin.
%
% Usage: \section{section name}
\renewcommand{\section}[1]{\pagebreak[3]%
    \hyphenpenalty=10000%
    \vspace{1.3\baselineskip}%
    \phantomsection\addcontentsline{toc}{section}{#1}%
    \noindent\llap{\scshape\smash{\parbox[t]{\marginparwidth}{\raggedright #1}}}%
    \vspace{-\baselineskip}\par}

%%% LISTS

% This macro alters a list by removing some of the space that follows the list
% (is used by lists below)
\newcommand*\fixendlist[1]{%
    \expandafter\let\csname preFixEndListend#1\expandafter\endcsname\csname end#1\endcsname
    \expandafter\def\csname end#1\endcsname{\csname preFixEndListend#1\endcsname\vspace{-0.6\baselineskip}}}

% These macros help ensure that items in outer-type lists do not get
% separated from the next line by a page break
% (they are used by lists below)
\let\originalItem\item
\newcommand*\fixouterlist[1]{%
    \expandafter\let\csname preFixOuterList#1\expandafter\endcsname\csname #1\endcsname
    \expandafter\def\csname #1\endcsname{\csname preFixOuterList#1\endcsname\let\oldItem\item\def\item{\pagebreak[2]\oldItem}}
    \expandafter\let\csname preFixOuterListend#1\expandafter\endcsname\csname end#1\endcsname
    \expandafter\def\csname end#1\endcsname{\let\item\oldItem\csname preFixOuterListend#1\endcsname}}
\newcommand*\fixinnerlist[1]{%
    \expandafter\let\csname preFixInnerList#1\expandafter\endcsname\csname #1\endcsname
    \expandafter\def\csname #1\endcsname{\let\oldItem\item\let\item\originalItem\csname preFixInnerList#1\endcsname}
    \expandafter\let\csname preFixInnerListend#1\expandafter\endcsname\csname end#1\endcsname
    \expandafter\def\csname end#1\endcsname{\csname preFixInnerListend#1\endcsname\let\item\oldItem}}

% An itemize-style list with lots of space between items
%
% Usage:
%   \begin{outerlist}
%       \item ...    % (or \item[] for no bullet)
%   \end{outerlist}
\newlist{outerlist}{itemize}{3}
    \setlist[outerlist]{label=\enskip\textbullet,leftmargin=*}
    \fixendlist{outerlist}
    \fixouterlist{outerlist}

% An environment IDENTICAL to outerlist that has better pre-list spacing
% when used as the first thing in a \section
%
% Usage:
%   \begin{lonelist}
%       \item ...    % (or \item[] for no bullet)
%   \end{lonelist}
\newlist{lonelist}{itemize}{3}
    \setlist[lonelist]{label=\enskip\textbullet,leftmargin=*,partopsep=0pt,topsep=0pt}
    \fixendlist{lonelist}
    \fixouterlist{lonelist}

% An itemize-style list with little space between items
%
% Usage:
%   \begin{innerlist}
%       \item ...    % (or \item[] for no bullet)
%   \end{innerlist}
\newlist{innerlist}{itemize}{3}
    \setlist[innerlist]{label=\enskip\textbullet,leftmargin=*,parsep=0pt,itemsep=0pt,topsep=0pt,partopsep=0pt}
    \fixinnerlist{innerlist}

% An environment IDENTICAL to innerlist that has better pre-list spacing
% when used as the first thing in a \section
%
% Usage:
%   \begin{loneinnerlist}
%       \item ...    % (or \item[] for no bullet)
%   \end{loneinnerlist}
\newlist{loneinnerlist}{itemize}{3}
    \setlist[loneinnerlist]{label=\enskip\textbullet,leftmargin=*,parsep=0pt,itemsep=0pt,topsep=0pt,partopsep=0pt}
    \fixendlist{loneinnerlist}
    \fixinnerlist{loneinnerlist}

%%% EXTRA SPACE

% To add some paragraph space between lines.
% This also tells LaTeX to preferably break a page on one of these gaps
% if there is a needed pagebreak nearby.
\newcommand{\blankline}{\quad\pagebreak[3]}
\newcommand{\halfblankline}{\quad\vspace{-0.5\baselineskip}\pagebreak[3]}

%%% FORMATTING MACROS

% Uses hyperref to link DOI
\newcommand\doilink[1]{\href{http://dx.doi.org/#1}{#1}}
\newcommand\doi[1]{doi:\doilink{#1}}

% For \url{SOME_URL}, links SOME_URL to the url SOME_URL
\providecommand*\url[1]{\href{#1}{#1}}
% Same as above, but pretty-prints SOME_URL in teletype fixed-width font
\renewcommand*\url[1]{\href{#1}{\texttt{#1}}}

% For \email{ADDRESS}, links ADDRESS to the url mailto:ADDRESS
\providecommand*\email[1]{\href{mailto:#1}{#1}}
% Same as above, but pretty-prints ADDRESS in teletype fixed-width font
%\renewcommand*\email[1]{\href{mailto:#1}{\texttt{#1}}}

%\providecommand\BibTeX{{\rm B\kern-.05em{\sc i\kern-.025em b}\kern-.08em
%    T\kern-.1667em\lower.7ex\hbox{E}\kern-.125emX}}
%\providecommand\BibTeX{{\rm B\kern-.05em{\sc i\kern-.025em b}\kern-.08em
%    \TeX}}
\providecommand\BibTeX{{B\kern-.05em{\sc i\kern-.025em b}\kern-.08em
    \TeX}}
\providecommand\Matlab{\textsc{Matlab}}

%%%%%%%%%%%%%%%%%%%%%%%% End Helper Commands %%%%%%%%%%%%%%%%%%%%%%%%%%%

%%%%%%%%%%%%%%%%%%%%%%%%% Begin CV Document %%%%%%%%%%%%%%%%%%%%%%%%%%%%



\begin{document}
\thispagestyle{plain}
\makeheading[\emph{Curriculum vitae}]{André Ambrósio Boechat}

\section{Informações de Contato}

% NOTE: Mind where the & separators and \\ breaks are in the following
%       table. Table is one row made up of three parboxes. The left
%       parbox has address info, the middle parbox has a vertical bar,
%       and the right parbox has phone and electronic contact
%       information.
%
% MACROS: \rcollength is the width of the right column of the table
%             (adjust it to your liking; default is 1.85in).
%         \spacewidth is width of area between left and right boxes.
%         \spacechar is character used to produce perforated vertical
%             boundary between boxes.
%
\newlength{\rcollength}\setlength{\rcollength}{3in}%
\newlength{\spacewidth}\setlength{\spacewidth}{10pt}
\newcommand\spacechar{$|$}
%
\begin{tabular}[t]{@{}p{\textwidth-\rcollength-\spacewidth}@{}p{\spacewidth}@{}p{\rcollength}}%

% Address box
\parbox{\textwidth-\rcollength-\spacewidth}{%
Florianópolis, SC\\
Brasil}

% Cheesy perforated vertical bar between boxes
% Shorten by removing \spacechar's
& \parbox{\spacewidth}{\centering \spacechar\\\spacechar\\\spacechar\\\spacechar\\\spacechar} &

% Non-snail-mail contact information
\parbox{\rcollength}{%
\textit{Celular:} +55-48-99091938 \\
\textit{E-mail:} \email{boechat107@gmail.com}\\
\textit{WWW:}
\href{https://sites.google.com/site/andreboechatpersonal}{sites.google.com/site/andreboechatpersonal}}

%\href{http://www.das.ufsc.br/~boechat}{www.das.ufsc.br/$\sim$boechat}}

\end{tabular}

%%
%% In modern CV's, it seems like ``Objective'' is frowned upon. Instead,
%% incorporate it into a well-constructed cover letter. The ``More
%% information'' can go at the end of the CV, but it should not distract
%% from the section giving references available to contact.
%%
%
% \section{Objective}
%
% Full-time position that allows for advanced research in electrical and
% computer engineering (communications, control, software, electronics,
% and sustainability), with a particular focus on complex distributed
% systems (i.e., modeling, analysis, design, and verification)
% \begin{innerlist}
%     \item For more information, see \url{http://www.tedpavlic.com/engjobsearch/}
% \end{innerlist}

% \section{Qualifications and Interests}
% 
% Advanced control systems, complex adaptive systems, agent-based
% modeling, hybrid dynamic systems, distributed algorithms, amorphous
% computing, autonomous systems and vehicles, networks, communications,
% verification, cooperation, optimization, game theory, parallel
% computation, robotics, analog electronics, behavioral ecology,
% bio-mimicry

%\section{Availability}
%
%\begin{loneinnerlist}
%    \item Start time is negotiable; may be possible to start immediately
%    \item Geographic location is flexible, but there is preference
%        for Tempe, AZ
%\end{loneinnerlist}
%
%\section{Security Clearance}
%
%Department of Defense Top Secret SCI with polygraph (expired: 2002)

% \section{Citizenship}
%
% USA


\section{Experiência}

\href{http://www.neoway.com.br/}{\textbf{Neoway Business Solutions}},
Florianópolis, SC, Brazil
\begin{outerlist}

\item[] \textit{Pesquisa e Desenvolvimento, programador}
    \hfill \textbf{Dezembro de 2011 --- Atualmente}
    \begin{innerlist}
    \item Emprego de técnicas de aprendizado de máquina e análises estatísticas para:
        \begin{innerlist}
        \item processamento e reconhecimento de imagens
        \item exploração de dados
        \end{innerlist}
    \end{innerlist}

\end{outerlist}
\halfblankline

\href{http://www.ufsc.br}{\textbf{Federal University of Santa Catarina}},
Florianópolis, SC, Brazil
\begin{outerlist}
    
\item[] \textit{R\&D Project \href{http://www.das.ufsc.br}{DAS-UFSC}/CENPES-Petrobras}
    \hfill \textbf{March 2010 to August 2012}
    \begin{innerlist}
    \item Manutenção baseada em Condição
    \item Processamento de Sinais
    \item Detecção de falhas
    \item Sistemas de monitoramento \textit{on-line} para sensores de poços de
        petróleo
    \end{innerlist}
\end{outerlist}
\halfblankline

\href{http://www.ufes.br}{\textbf{Federal University of Espírito Santo}},
Vitória, ES, Brazil
\begin{outerlist}
    
\item[] Estagiário no Laboratório de Controle Inteligente
    \hfill \textbf{September 2008 to February 2010}
    \begin{innerlist}
    \item Detecção de oscilações em malhas de controle
    \item Detecção de falhas em válvulas de controle
    \item LabView
    \end{innerlist}

\item[] Programa Institucional de Iniciação Científica
    \hfill \textbf{July 2007 to October 2008}
    \begin{innerlist}
    \item Simulação numérica de reservatórios de petróleo usando LibMesh
    \end{innerlist}

\item[] Intern at \href{http://www.ufes.br/~pet}{PET Engenharia de Computação}
    \hfill \textbf{March 2006 to August 2008}
    \begin{innerlist}
    \item Envolvimento em atividades educacionais e de pesquisa
    \item Organização de eventos científicos e estudantis
    \item Participação de conferências locais e nacionais
    \end{innerlist}

\end{outerlist}


%\section{Refereed Journal Publications}
%
%\begin{bibenum}
%    \item Pavlic, T.P., and K.M.~Passino. Generalizing foraging theory
%        for analysis and design. \emph{The International Journal of
%        Robotics Research [Special Issue on Stochasticity in Robotics
%        and Bio-Systems Part 1]}. 30(5):505--523. 2011.\\
%        \doi{10.1177/0278364910396551}
%
%    \item Pavlic, T.P., and K.M.~Passino. The sunk-cost effect as an
%        optimal rate-maximizing behavior. \emph{Acta Biotheoretica},
%        59(1):53--66. 2011.\\
%        \doi{10.1007/s10441-010-9107-8}
%
%    \item Pavlic, T.P., and K.M.~Passino. When rate maximization is
%        impulsive. \emph{Behavioral Ecology and Sociobiology},
%        64(8):1255--1265. August 2010.\\
%        \doi{10.1007/s00265-010-0940-1}
%
%    \item Pavlic, T.P., and K.M.~Passino. Foraging theory for autonomous
%        vehicle speed choice. \emph{Engineering Applications of
%        Artificial Intelligence}, 22(3):482--489, April
%        2009.\\
%        \doi{10.1016/j.engappai.2008.10.017}
%\end{bibenum}
%
%\section{Submitted Journal Publications}
%
%\begin{bibenum}
%    \item Pavlic, T.P., and K.M.~Passino. Cooperative task processing.
%       \emph{IEEE Transactions on Systems, Man, and Cybernetics}. 2012.
%        Submitted.
%\end{bibenum}

% Add a little space to nudge next ``Conference Publications'' marginpar
% down to make room for tall ``Submitted Journal Publications''
% marginpar. If there are enough submitted journal publications, this
% space will not be needed (and should be removed).
%\vspace{0.1in}
%
%\section{Conference Publications}

%\begin{bibenum}
%    \item Pavlic, T.P., P.A.G.~Sivilotti, A.D.~Weide, and B.W.~Weide.
%        Verification of Smooth and Close Collision-Free Cruise Control.
%        In: \emph{Proceedings of the 2011 Symposium on Control and
%        Modeling Cyber-Physical Systems}, October 20--21, 2011. Poster
%        abstract.
%
%    \item {\"{O}}zg{\"{u}}ner, {\"{U}}., A.~Krishnamurthy,
%        F.~{\"{O}}zg{\"{u}}ner, K.~Redmill, P.~Sivilotti, B.~Weide,
%        and T.~Pavlic.
%        CPS: Autonomous driving in urban environments.
%        In: \emph{Proceedings of the 2011 NSF CPS PI Meeting},
%        August 1--2, 2011. Poster abstract.
%
%    \item Pavlic, T.P. Stigmergic memory for real-time primal-space
%        distributed optimization under constraints.
%        In: \emph{Proceedings of the 50th IEEE Conference on Decision
%        and Control and European Control Conference (CDC-ECC~2011)},
%        December 12--15, 2011. Submitted.
%
%    \item Pavlic, T.P., and K.M.~Passino. Cooperative task-processing
%        networks. In: \emph{Proceedings of the Second International
%        Workshop on Networks of Cooperating Objects (CONET~2011)},
%        April 11, 2011.
%
%    \item Pavlic, T.P., and K.M.~Passino. Cooperative task
%        processing. In: \emph{Proceedings of the ICAM 2009 Symposium:
%        Emergence in Physical, Biological, and Social Systems IV},
%        November 13, 2009. Poster abstract.
%
%    \item Freuler, R.J., M.J.~Hoffmann, T.P.~Pavlic, J.M.~Beams,
%        J.P.~Radigan, P.K.~Dutta, J.T.~Demel, and E.D.~Justen.
%        Experiences with a Comprehensive Freshman Hands-On Course~--
%        Designing, Building, and Testing Small Autonomous Robots. In:
%        \emph{Proceedings of the 2003 American Society for Engineering
%        Education Annual Conference \& Exposition}, 2003.
%\end{bibenum}


\section{Educação}

\href{http://www.ufsc.br}{\textbf{Federal University of Santa Catarina}},
Florianópolis, SC, Brazil
\begin{outerlist}

\item[] M.S.,
        \href{http://www.pgeas.ufsc.br/}
             {Automation and System Engineering}, 2010 -- 2012 (expected)
        \begin{innerlist}
        \item Topics: signal processing, fault detection and isolation, empirical modeling
            based on historical data, sensor drift
        \item R\&D scholarship
        \item 1 paper published at international workshop on automatic control of offshore
            oil and gas production
        \end{innerlist}
\end{outerlist}

\blankline

\href{http://www.ufes.br}{\textbf{Federal University of Espírito Santo}},
Vitória, ES, Brazil
\begin{outerlist}

\item[] B.S.,
        \href{http://www.inf.ufes.br/}
             {Computer Engineering}, 2005 -- 2009
        \begin{innerlist}
        \item \href{http://www.inf.ufes.br/~pet}{PET} scholarship 
        \item Program of Scientific Initiation (PIVIC)
        \item Participation in 2 research projects
        \item 1 paper published and presented at international congress on computational
            methods in engineering
        \item 1 paper published at international conference on industry applications
        \end{innerlist}

\end{outerlist}

\section{Publicações}

\begin{bibenum}

    \item \href{http://www.ifac-papersonline.net/Detailed/52627.html}{Boechat, Andre A.;
            Moreno, Ubirajara F.; Haramura, Decio, Jr., “On-Line Calibration Monitoring
            System Based on Data-Driven Model for Oil Well Sensors,” in Proceedings of the
            IFAC Workshop on Automatic Control in Offshore Oil and Gas Production,
        Trondheim, Norway, 2012, pp. 269–274.}

    \item BOECHAT, A. A., MORENO, U. F. and TEIXEIRA, A. F. . PROPOSTA DE UMA ARQUITETURA
        DE MONITORAMENTO BASEADO NA CONDIÇÃO PARA APLICAÇÃO EM UNIDADES MARÍTIMAS DE
        PRODUÇÃO. VI Congresso Rio Automação, 2011, Rio de Janeiro. VI Congresso Rio
        Automação, 2011.

    \item
        \href{http://www.labplan.ufsc.br/congressos/Induscon\%202010/fscommand/web/docs/I0187.pdf}{MUNARO,
            C. J. and BOECHAT, A. A. . Ambiente computacional baseado em Labview para
            detecção de oscilações em malhas de controle. IX Induscon - 2010 9th IEEE/IAS
        International Conference on Industry Applications, 2010, São Paulo.}

    \item BOECHAT, A. A. and VALLI, A. M. P. . Funcionalidades da biblioteca LibMesh na
        resolução de escoamento de fluidos pelo método dos elementos finitos. 
        Congresso Ibero-Latino-Americano de Métodos Computacionais em Engenharia, 2009,
        Armação de Búzios. CILAMCE 2009 - Congresso Ibero-Latino-Americano de Métodos
        Computacionais em Engenharia, 2009.

    %\item Pavlic, T.P. \emph{Design and Analysis of Optimal
    %    Task-Processing Agents}. PhD thesis, The Ohio State University,
    %    Columbus, OH, 2010.

    %\item Pavlic, T.P. \emph{Optimal Foraging Theory Revisited}.
    %    Master's thesis, The Ohio State University, Columbus, OH, 2007.
\end{bibenum}

%\section{Books in Preparation}
%
%\begin{bibenum}
%    \item Pavlic, T.P., B.W.~Andrews, K.M.~Passino, and T.A.~Waite.
%        \emph{Foraging Theory for Engineering}.
%\end{bibenum}
%
%\section{Papers in Preparation}
%
%\begin{bibenum}
%    \item Pavlic, T.P., K.M.~Passino. Distributed optimization under
%        constraints: Pareto-optimal intelligent lighting.
%
%    \item Pavlic, T.P. The ideal free distribution as degenerate form of
%        nutrient-constrained optimization.
%\end{bibenum}
%
%\section{Grants}
%
%\begin{bibsection}
%
%    \item Senior staff,
%        ``Autonomous Driving in Mixed-Traffic Urban Environments'',
%        NSF,\linebreak[4]
%        \href{http://www.nsf.gov/awardsearch/showAward.do?AwardNumber=0931669}{ECCS-0931669},
%        \$1,499,833, September 1,~2009~-- August 31,~2012.
%
%    \item Senior staff,
%        ``Cooperative LED Arrays for Preference-Adaptive Lighting in
%        Smart Buildings'',
%        NSF,
%        EFRI-SEED preliminary proposal, 2009.
%
%\end{bibsection}

% \section{Referee Service}
% 
% \begin{loneinnerlist}
%     \item \emph{49\textsuperscript{th} Annual Conference on Decision and Control}
%     \item \emph{Bioinspiration \& Biomimetics}
%     \item \emph{Behavioral Ecology}
%     \item \emph{IEEE Transactions on Signal Processing}
%     \item \emph{The International Journal of Robotics Research}
%     \item \emph{Swarm and Evolutionary Computation}
%     \item \emph{IEEE Transactions on Control Systems Technology}
%     \item \emph{International Journal of Control}
% \end{loneinnerlist}
% 
% \section{Conference Service}
% 
% \begin{bibsection}
%     \item Organizer/Associate Editor for invited session: ``Correctness
%         by Verification and Design'', 14\textsuperscript{th} IEEE
%         Conference on Intelligent Transportation Systems~(ITSC~2011),
%         Washington, DC, October 5--7, 2011.
% \end{bibsection}
% 
% \section{Student Advising}
% 
% \begin{bibsection}
% 
%     \item \textbf{Cory Henderson}, \textbf{James O'Donnell},
%         \textbf{Ian Neack}, and \textbf{Patrick Whewell}.\\
%         Undergraduate students in Electrical and Computer Engineering.
%         Group design project on retrofittable
%         vehicle-to-vehicle communications system for
%         adaptive-cruise-control in mixed-traffic environments. 2012.
% 
%     \item \textbf{Manas Agrawal}.
%         Graduate student in Computer Science and Engineering.
%         Software verification and model checking applied to railroad
%         safety problems. 2012.
% 
%     \item \textbf{Sri Prathyusha Peddi}.
%         Graduate student in Computer Science and Engineering.
%         Software verification applied to adaptive cruise
%         control and instrumented intersection signal timing. 2011--2012.
% 
%     \item \textbf{Jaeyong Park}.
%         Graduate student in Electrical and Computer Engineering.
%         Provably correct on-line control synthesis for
%         autonomous vehicles with hybrid dynamics. 2011--2012.
% 
% \end{bibsection}
% 
% \section{Teaching Experience}
% 
% \href{http://www.osu.edu}{\textbf{The Ohio State University}},
% Columbus, Ohio USA
% \begin{outerlist}
% 
% \item[] \textit{Instructor}%
%     \hfill \textbf{March~2012~-- August~2012}
%     \begin{innerlist}
%         \item Instructor for ECE~683: Undergraduate Design Project
%             %
%             \begin{innerlist}
%                 \item Students designed retrofitable vehicle-to-vehicle
%                     communications system to aid in the development of
%                     verifiably safe adaptive cruise control.
%                 \item Design project folded into larger research project
%                     on autonomous vehicles in mixed-traffic urban
%                     environments.
%             \end{innerlist}
%     \end{innerlist}
% 
% \item[] \textit{Teaching Assistant}%
%     \hfill \textbf{September 2007~-- August 2009}\\
%     (sample graded material and student evaluations available upon
%     request)
%     \begin{innerlist}
%         \item Instructor for ECE~327: Electronic Devices and Circuits Laboratory I
%         \begin{innerlist}
%             \item Autumn~2007, Winter~(2) and Spring~2008~(2),
%                 Winter~(2) and Summer~2009
% 
%             % \item Sample student evaluations available upon request.
% 
%             \item Responsible for 1-hour lecture and supervision of
%                 3-hour laboratory. Students design and implement
%                 infrared modem and 8-ohm speaker driver.
% 
%             \item Authored hundreds of pages of course material
%                 archived at\\
%                 \url{http://www.tedpavlic.com/teaching/osu/ece327}.
%         \end{innerlist}
% 
%         \halfblankline
% 
%         \item Grader for ECE~481 Ethics in Electrical and Computer Engineering
%         \begin{innerlist}
%             \item Autumn~2007 and Autumn~2008
%         \end{innerlist}
% 
%         \halfblankline
% 
%         \item Instructor for ECE~209: Circuits and Electronics
%             Laboratory
%         \begin{innerlist}
%             \item Autumn~2008
% 
%             % \item Sample student evaluations available upon request.
% 
%             \item Responsible for lecture and supervision of basic
%                 electronics laboratory.
% 
%             \item Authored material at
%                 \url{http://www.tedpavlic.com/teaching/osu/ece209}.
%         \end{innerlist}
% 
%         \halfblankline
% 
%         \item Instructor for ECE~557: Control, Signals, and Systems
%             Laboratory
%         \begin{innerlist}
%             \item Summer~2008~(2~sections) and Summer~2009
% 
%             % \item Sample student evaluations available upon request.
% 
%             \item Responsible for lecture and supervision of laboratory.
%                 Students used
%                 \href{http://www.mathworks.com/products/simulink/}{Simulink}
%                 and \href{http://www.dspaceinc.com/}{dSPACE} RTI1104
%                 units for linear system control design.
% 
%             \item Authored material at
%                 \url{http://www.tedpavlic.com/teaching/osu/ece557}.
%         \end{innerlist}
% 
%         \halfblankline
% 
%         \item Lab Instructor for ECE~758: Control Systems Implementation
%             Laboratory
%         \begin{innerlist}
%             \item Spring~2009~(2~sections)
% 
%             % \item Sample student evaluations available upon request
% 
%             \item Responsible for lecture and supervision of laboratory.
%                 Graduate and senior undergraduate students used
%                 \href{http://www.mathworks.com/products/simulink/}{Simulink},
%                 with \href{http://www.dspaceinc.com/}{dSPACE} RTI1104
%                 units for analysis of and advanced control implementation
%                 for linear and non-linear systems.
% 
%             \item Authored material at
%                 \url{http://www.tedpavlic.com/teaching/osu/ece758}.
%         \end{innerlist}
%     \end{innerlist}
% 
% \item[] \href{http://www.nsfgk12.org/}
%         {\emph{National Science Foundation GK-12 Fellow}}
%         \hfill \textbf{September 2006~-- October 2007}
% \begin{innerlist}
%     \item[] Developed, implemented, and evaluated daily inquiry-based
%         fourth-grade science lessons for a local inner-city public
%         school class.
% \end{innerlist}
% 
% \item[] \textit{Instructor}%
%         \hfill \textbf{March 2002~-- June 2004}
% \begin{innerlist}
% \item Member of \href{http://feh.eng.ohio-state.edu/}
%                      {Fundamentals of Engineering for Honors}
%       instructional team.
% \item Special graduate teaching appointment as undergraduate.
% \item Lectured weekly engineering laboratory for ENG~H191,
%         H192, and~H193.
% \item Trained in-class undergraduate teaching assistants in laboratory
%         procedure.
% \item Graded weekly lab reports and provided laboratory exams.
% \end{innerlist}
% 
% \item[] \textit{Teaching Assistant}%
%         \hfill \textbf{September 2000~-- March 2002}
% \begin{innerlist}
% \item Assisted \href{http://feh.eng.ohio-state.edu/}
%                     {Fundamentals of Engineering for Honors}
%       instructional team.
% \item Provided support to first-year engineering students
%         (ENG~H191, H192, and~H193).
% \item Graded daily assignments on programming and drafting.
% \item Developed on-line journal submission and report system for Physics
%         Education Research Group~(PERG).
% \end{innerlist}
% 
% \item[] \textit{Undergraduate Researcher}%
%         \hfill \textbf{September 2000~-- March 2002}
% \begin{innerlist}
% \item Participated in the
%         \href{http://www.cse.ohio-state.edu/europa/}{Europa
%         Undergraduate Research Forum}, a part of the
%         \href{http://www.cse.ohio-state.edu/rsrg/}{Reusable Software
%         Research Group}.
% \item Studied component-based software engineering undergraduate
%         pedagogy.
% \item Researched changes to RESOLVE/C++ implementation for ANSI
%         compliance.
% \end{innerlist}
% 
% \item[] \textit{Grader}%
%         \hfill \textbf{September--December 2001}
% \begin{innerlist}
% \item Graded daily electromagnetics assignments (ECE~311).
% \end{innerlist}
% \end{outerlist}
% 
% \section{Professional Memberships}
% 
% Institute for Electrical and Electronics Engineers~(IEEE), Member,
% 2002--present
% %
% \begin{innerlist}
% \item IEEE Control Systems Society (2004--present)
% \item IEEE Computer Society (2009--present)
% \item IEEE Intelligent Transportation Systems Society (2011--present)
% \item IEEE Systems, Man, and Cybernetics Society (2011--present)
% \item IEEE Robotics and Automation Society (2011--present)
% \end{innerlist}
% 
% \halfblankline
% 
% Animal Behavior Society, Member, 2011--present
% 
% \halfblankline
% 
% Society for Mathematical Biology, Member, 2012--present
% 
% \section{Service}
% 
% Recent contributor to several open-source software projects, including:
% \begin{innerlist}
%     \item \href{http://vim-latex.sourceforge.net/}{Vim-LaTeX} suite
%     \item \href{http://vimperator.org}{Vimperator} and
%         \href{http://dactyl.sourceforge.net/pentadactyl/index}{Pentadactyl}
%         Firefox extensions
%     \item \href{http://git-scm.com}{Git} distributed version control
%         system
%     \item \href{http://www.selenic.com/mercurial/}{Mercurial} distributed version control
%         system
%     \item Personal projects archived at
%         \url{http://hg.tedpavlic.com/}
% \end{innerlist}
% 
% \halfblankline
% 
% Frequent contributor to \href{http://www.wikipedia.org/}{Wikipedia}.
% %
% \begin{innerlist}
%     \item Significant contributions to articles on control theory,
%         electronics, and signals and systems.
% \end{innerlist}
% 
% \halfblankline
% 
% Contributor to \href{http://www.quora.com/}{Quora}.
% %
% \begin{innerlist}
%     \item Contributions to articles on thermodynamics, chaos theory,
%         electronics, and evolutionary biology.
% \end{innerlist}
% 
% \halfblankline
% 
% \href{http://www.osufirst.org/}{OSU FIRST Robotics Team},
% \href{http://www.osu.edu}{The Ohio State University}, 2000--2004
% \begin{innerlist}
% \item Introduced middle school and high school students to science and
%         technology by participating with them in national robotics
%         competitions.
% \item Led 2002 team to regional silver medal
%         \href{http://www.firstwiki.org/Engineering_Inspiration_Award}
%              {\emph{Engineering Inspiration Award}}.
% \item \emph{Lead Team Mentor}, 2002--2004
% \item \emph{Component Design Team Lead Mentor}, 2001--2002
% \end{innerlist}
% 
% \halfblankline
% 
% Director of Computers,
% \href{http://ec.osu.edu/}{Engineers' Council},
% \href{http://www.osu.edu/}{The Ohio State University}, 2002
% 
% \halfblankline
% 
% \href{http://www.linuxvirtualserver.org/}
%      {Linux Virtual Server Project}, 1999--2000
% \begin{innerlist}
% \item Early member of the team that formed the open-source project that
%         is now an important load balancing solution for the Linux
%         software platform.
% \end{innerlist}
% 
% \halfblankline
% 
% \href{http://www.gcfn.org/}
%      {Greater Columbus Free-Net}, 1995--1997
% \begin{innerlist}
% \item Provided technical support services.
% \end{innerlist}
% 
% \halfblankline
% 
% CompuTeen Bulletin Board System, 1993--1995
% \begin{innerlist}
% \item Administrated dial-up bulletin board system.
% \item Founded and administrated TeenLiNK, an international electronic
%         mail network that spread through the United States, Canada, and
%         Australia and delivered mail over a series of electronic dial-up
%         drop offs.
% \end{innerlist}
% 
% 

\section{Línguas}

Português
\begin{innerlist}
\item Nativo
\item C2 CEFR
\end{innerlist}

\halfblankline

Inglês
\begin{innerlist}
\item Leitura: muito bom
\item Escrita: bom
\item Conversação: avançada
\item aproximadamente B2 CEFR
\end{innerlist}

\halfblankline

Espanhol
\begin{innerlist}
\item Leitura: bom
\item Escrita: básico
\item Conversação: básico
\item aproximadamente A2 CEFR
\end{innerlist}

\section{Informações Adicionais}

GitHub: \href{https://github.com/boechat107}{github.com/boechat107}

% \section{Wage expectation}
% 
% R\$ 6000,00 for 40 hours/week.



\end{document}
%%%%%%%%%%%%%%%%%%%%%%%%%% End CV Document %%%%%%%%%%%%%%%%%%%%%%%%%%%%%

%----------------------------------------------------------------------%
% The following is copyright and licensing information for
% redistribution of this LaTeX source code; it also includes a liability
% statement. If this source code is not being redistributed to others,
% it may be omitted. It has no effect on the function of the above code.
%----------------------------------------------------------------------%
% Copyright (c) 2007, 2008, 2009, 2010, 2011 by Theodore P. Pavlic
%
% Unless otherwise expressly stated, this work is licensed under the
% Creative Commons Attribution-Noncommercial 3.0 United States License. To
% view a copy of this license, visit
% http://creativecommons.org/licenses/by-nc/3.0/us/ or send a letter to
% Creative Commons, 171 Second Street, Suite 300, San Francisco,
% California, 94105, USA.
%
% THE SOFTWARE IS PROVIDED "AS IS", WITHOUT WARRANTY OF ANY KIND, EXPRESS
% OR IMPLIED, INCLUDING BUT NOT LIMITED TO THE WARRANTIES OF
% MERCHANTABILITY, FITNESS FOR A PARTICULAR PURPOSE AND NONINFRINGEMENT.
% IN NO EVENT SHALL THE AUTHORS OR COPYRIGHT HOLDERS BE LIABLE FOR ANY
% CLAIM, DAMAGES OR OTHER LIABILITY, WHETHER IN AN ACTION OF CONTRACT,
% TORT OR OTHERWISE, ARISING FROM, OUT OF OR IN CONNECTION WITH THE
% SOFTWARE OR THE USE OR OTHER DEALINGS IN THE SOFTWARE.
%----------------------------------------------------------------------%

